\documentclass{beamer}
\usepackage{etoolbox}\newtoggle{printable}\togglefalse{printable}
\usetheme{Copenhagen}
\usecolortheme{beaver}
\usepackage{listings}
\usepackage{algpseudocode}
\pdfmapfile{+sansmathaccent.map}
\graphicspath{ {Images/} }

\title{Comp310 Demake Proposal}
\author{Alastair Rayner}
\date{\today}

\begin{document}

\maketitle


\begin{frame}{About the DeMake}
	  For this Demake project i will be creating StarCitizen for the 6502 CPU on the NES. \pause
	
	 This presentation will aim to address the Concept and Technical Feasibility of the game. \pause
	
\end{frame}


\begin{frame}{Concept}		
	\textbf{Game Concept} \pause
		\begin{itemize}
			\item The concept of my demake is a 2D space shooter, similar to Galaga but based on the modern game Star Citizen.  \pause
			\item The game will have a player ship at the bottom of the screen and will be able to fire projectiles at enemies that spawn at the top of the screen.\pause
			
			\item The player will loose health if hit by projectiles, but if the player kills enemy ships their score will increase.  \pause

			\item The aim of the game is to get the best score. \pause 	
		\end{itemize}
\end{frame}

\begin{frame}{Key Mechanic}		
	\textbf{Key Mechanic} \pause
		\begin{itemize}
			\item The key mechanic of which this demake will be based around firing projectiles at enemy ships and dodging enemies bullets and ships.  \pause

		\end{itemize}
\end{frame}

\begin{frame}{Technical Feasibility}		
	\textbf{Technical Feasibility} \pause
		\begin{itemize}
			\item This style of game has been done a few times on legacy game systems, so it should be very feasible.  \pause
			\item The game will have a detailed player sprite, consisting of 4 8x8 sprites, and the enemies will be just 8x8 sprites, with the exception of any bosses.  \pause
		\end{itemize}
\end{frame}


\end{document}
